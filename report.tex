\documentclass[a4paper, 11pt]{article}
\usepackage{comment} % enables the use of multi-line comments (\ifx \fi) 
\usepackage{lipsum} %This package just generates Lorem Ipsum filler text. 
\usepackage{fullpage} % changes the margin
\usepackage{booktabs}
\usepackage{graphicx}
\usepackage{float}
\usepackage[brazilian]{babel}
\usepackage[utf8]{inputenc}
\usepackage[T1]{fontenc}
\usepackage[dvipsnames]{xcolor}
\usepackage{hyperref}
\usepackage{listings}
\usepackage{color}

\definecolor{codegreen}{rgb}{0,0.6,0}
\definecolor{codegray}{rgb}{0.5,0.5,0.5}
\definecolor{codepurple}{rgb}{0.58,0,0.82}
\definecolor{backcolour}{rgb}{0.98,0.98,0.98}
 
\lstdefinestyle{mystyle}{
    backgroundcolor=\color{backcolour},   
    commentstyle=\color{blue},
    keywordstyle=\color{codegreen},
    numberstyle=\tiny\color{codegray},
    stringstyle=\color{codepurple},
    basicstyle=\footnotesize,
    breakatwhitespace=false,         
    breaklines=true,                 
    captionpos=b,                    
    keepspaces=true,                 
    numbers=left,                    
    numbersep=5pt,                  
    showspaces=false,                
    showstringspaces=false,
    showtabs=false,                  
    tabsize=2
}

\lstset{style=mystyle}

\begin{document}
%Header-Make sure you update this information!!!!
\noindent
\large\textbf{INF01145 - Fundamentos de Bancos de Dados} \hfill \\
\normalsize Relatorio 1 \hfill Ana C. Pagnoncelli, Rafael B. Audibert \\
Profª. Karin Becker \hfill 02/07/2019 \\

%% ===================================
\section*{Enunciado do trabalho}

O trabalho prático da disciplina deve versar sobre o projeto e uso de uma base de dados para um sistema de informação (SI), a
ser modelado e implantado em computador. O trabalho envolve a modelagem e o projeto da base de dados com o uso de
ferramentas de modelagem, bem como criação, instanciação e manipulação em um modelo relacional.

Esse relatório se refere à primeira parte do trabalho, dizendo respeito à implementação de um projeto conceitual da Base de Dados e sua subsequente implementação em SGBD relacional.

%% ===================================
\section*{Universo de Discurso}
O universo de discurso é baseado no site/aplicativo Google Play\cite{GooglePlay}, que é um serviço de distribuição digital de aplicativos, jogos, filmes, programas de televisão, músicas e livros, desenvolvido e operado pela Google\cite{Google}. Ela é a loja oficial de aplicativos para o sistema operacional Android, além de fornecer conteúdo digital.

O universo é composto por:
\begin{itemize}
    \item \textbf{Usuário:} Deve possuir nome, senha, cpf, data de nascimento e email (único). O usuário é responsável pela maior parte das ações no aplicativo, sendo elas:
    \begin{itemize}
        \item Comprar Itens (livros, álbuns de música, aplicativos e filmes): Pode-se fazer o download de quantos e quaisquer itens que quiser, porém nunca o mesmo item duas vezes. Existem itens que são pagos, e portanto, suas compras só podem ser feitas por cartão e aceitas se o usuário tiver um cartão cadastrado e pagar o preço referente ao item.\\
        Depois de comprado o item, o download do mesmo é feito, e a informação de que o download foi feito é salva. Fazendo com
        que o item seja eternamente do usuário, ou seja, se ele quiser excluir do celular, terá a possibilidade de baixar novamente quando quiser, sem precisar pagar novamente.
        \\
        \item Adicionar Itens na Lista de Desejos: Quando o usuário encontra um item que gosta e tem intenção de comprar no futuro, adiciona na sua lista de desejos, que o possibilita voltar a qualquer momento e achar o item de forma rápida. Podem ser adicionados quantos e quaisquer itens que desejar, porém apenas uma vez cada um deles.
        \\
        \item Revisão de Itens: Usuários podem dar sua opinião sobre itens, tanto os que já foram usados na plataforma, quanto vistos em outros lugares. Cabe ao bom senso do usuário comentar apenas em itens que já usou. \\
        Cada usuário só pode fazer uma revisão por item, mas pode revisar quantos itens quiser. Uma revisão é composta por uma nota (de 0 a 5), uma data e opcionalmente um breve comentário.
        \\
        \item Adicionar cartões: Cartões são usados para realizar compras. O usuário pode adicionar quantos desejar. O cartão não necessariamente precisa estar em seu nome, portanto é necessário que guardemos o nome impresso no cartão. Porém não pode adicionar novamente o mesmo cartão.
        \\
        \item Baixar Itens: O download é ligado com a compra (mesmo que seja de um item gratuito), assim cada vez que é feita uma compra, um download é realizado. O mesmo guarda a data e a informação de que o download foi feito, uma forma de demonstrar que o usuário possui o item baixado, para que futuramente por mais que o exclua, continue tendo permissão para baixar novamente, sem que seja necessário pagar pelo mesmo novamente.
    \end{itemize}.
    \item \textbf{Cartão:} Definido por número (único), data de vencimento e usuário que é responsável pelo mesmo (único), além do nome impresso em cima do cartão, já que, como descrito acima, o dono do cartão não necessariamente precisa ser o usuário dono da conta. Sendo o usuário com um ou mais cartões, responsável pela compra de itens (que possuem um valor maior que zero). Não existem cartões que não possuam um dono, e cada cartão pode pertencer a somente um usuário.
    \item \textbf{Item:} Pode ser livro, aplicativo, álbum de música ou filme, e é o objeto de consumo do usuário e base da plataforma.\\
    Definido por preço (sendo o item considerado gratuito quando esse campo tiver o valor 0, ou não estiver preenchido, já que é opcional), nome, data de lançamento, resumo e código (único). Além disso o item pode entrar na promoção, sendo necessário guardar o valor dele com desconto e sua data limite de promoção (que pode ser nula caso o item não esteja em promoção). Caso a data atual seja menor do que a data limite da promoção, entende-se que o item está em promoção, portanto o valor do item, é o valor com desconto. Não existe interesse do sistema em armazenas um histórico de promoções\\
    O item é ligado em quase todas as ações disponíveis do usuário (menos adicionar cartões), sendo dividido em quatro tipos:
    \begin{itemize}
        \item Aplicativo: Além de todos os atributos do item possui versão, tamanho (que pode ser, no máximo 1TB) e um desenvolvedor. 
         \begin{itemize}
            \item Desenvolvedor: Possui endereço, e-mail, senha e uma quantidade $n$ de aplicativos criados. É um tipo de conta, feita para criadores de aplicativos, onde é considerado um desenvolvedor qualquer pessoa que criou uma conta deste tipo, ou seja, não é necessário ter um aplicativo para ser um desenvolvedor.
         \end{itemize}.
        \item Livro: Além de todos os atributos do item possui ISBN\cite{ISBN} (único), número de páginas, um único autor e uma única linguagem. Diferentes linguagens do mesmo livro, correspondem a diferentes itens.
        \begin{itemize}
            \item Autor: Autores são pessoas que produziram algum conteúdo que está disponibilizado no sistema. Autores de livros também podem ter produzido algum álbum, ou ter participado de algum filme. Sempre que um livro é cadastrado, é necessário autor já existente, ou, caso não exista criar um novo.\\
            \item Linguagem: Liguagens são pré-definidas, para que possam, além de controlar as linguagens de livros existentes, ajudar em buscas específicas.\\
            Como linguagens são definidas antes de livros, existem linguagens sem livros cadastrados com elas, mas as mesmas podem ter até $n$ livros. Livros sempre tem apenas uma linguagem. Inicialmente, o sistema possuirá as as linguagens descritas na Tabela 1, porém, futuramente poderão ser adicionadas novas linguagens ao sistema, de acordo com a expansão do mesmo.
        \end{itemize}.
        \item Filme: Além de todos os atributos do item possui duração, linguagens e integrantes.
        \begin{itemize}
            \item Linguagem: Possui o mesmo objetivo da linguagem do livro, com a única diferença de que quando se relaciona com o filme pode ser de dois tipos, áudio ou legenda. Pode existir várias linguagens de áudio e legenda para um mesmo filme, porém não podem se repetir no mesmo filme (ex: duas vezes áudio em português).
            \item Integrantes: Compostas por nome, são pessoas que participaram da produção do filme, com certas funções. Uma mesma pessoa pode ter diferentes funções em diferentes filmes, porém não pode desempenhar a mesma função duas vezes no mesmo filme. Esses integrantes também podem ter escrito algum livro, ou lançado um álbum.
        \end{itemize}.
        \item Álbum: Além de todos os atributos do item possui faixas de música e artista.
        \begin{itemize}
            \item Artista: É o cantor ou grupo que criou o álbum, que pode também ter escrito algum livro, ou participado de algum filme. Tem um nome único e pode possuir quantos álbuns quiser, porém um álbum só tem um artista que o compôs.
            \item Faixa de Música: São as músicas que compõem o álbum, elas tem um nome único e uma duração. Cada música só está presente em um álbum, porém um álbum tem várias músicas. Entende-se que o artista responsável por essa música, é o mesmo responsável pelo álbum ao qual essa música pertence.
        \end{itemize}
    \end{itemize}.
    Além desta divisão de tipos, os itens também tem categorias e anexos, que são:
    \begin{itemize}
        \item Categoria: Nada mais que os gêneros de um item (ex: ação para filmes, pop para músicas...). Gêneros são pré-definidos e um item só pode ser cadatrado com gêneros que já existem. Além disso, itens só podem ter gêneros que fazem parte do seu tipo, por exemplo, um álbum não pode ter gênero ação.\\
        Dado isso, um item pode ter quantos gêneros quiser, desde que não repita o mesmo, seja do seu tipo e esteja contido na tabela de gêneros, inicialmente definida pelos valores da tabela 2, porém podendo ser adicionadas novas categorias no futuro.
        \item Anexos: Itens podem ter fotos e vídeos anexados para demonstrar seu conteúdo. Sendo necessário que para cada anexo adicionado, tenha caminho em disco de onde o arquivo foi armazenado (único), tipo de arquivo, nome e extensão.
    \end{itemize}
\end{itemize}

\begin{table}[H]
\centering
\begin{tabular}{@{}llll@{}}
\toprule
\multicolumn{4}{c}{\textbf{Linguagens}} \\ \midrule
\multicolumn{1}{l|}{Português}& \multicolumn{1}{l|}{Inglês} & \multicolumn{1}{l|}{\begin{tabular}[c]{@{}l@{}}Espanhol\end{tabular}} &
Alemão  \\[5pt]
\multicolumn{1}{l|}{Italiano} & 
\multicolumn{1}{l|}{Francês} & \multicolumn{1}{l|}{Russo} &
Japones \\[5pt]

\multicolumn{1}{l|}{Islândes} & 
\multicolumn{1}{l|}{Norueguês} & \multicolumn{1}{l|}{Sueco} &
Coreano \\[5pt]
 \bottomrule
\end{tabular}
\caption{Linguagens possíveis no sistema}
\end{table}

\newpage
\begin{table}[H]
\centering
\resizebox{\textwidth}{!}{%
\begin{tabular}{@{}llll@{}}
\toprule
\multicolumn{4}{c}{\textbf{Categorias}} \\ \midrule
\multicolumn{1}{l|}{\textit{App}} & \multicolumn{1}{l|}{\textit{Filme}} &
\multicolumn{1}{l|}{\textit{Livro}} & \textit{Álbum} \\ \midrule

\multicolumn{1}{l|}{Arte e Design}& \multicolumn{1}{l|}{Ação e Aventura} & \multicolumn{1}{l|}{\begin{tabular}[c]{@{}l@{}}Arte\end{tabular}} & Alternativa  \\[5pt]

\multicolumn{1}{l|}{Carro e Veículos} & 
\multicolumn{1}{l|}{Animação} & \multicolumn{1}{l|}{Terror} & Blues \\[5pt]

\multicolumn{1}{l|}{Beleza} & 
\multicolumn{1}{l|}{Comédia} & \multicolumn{1}{l|}{Biografias} & MPB \\[5pt]

\multicolumn{1}{l|}{Esportes} & 
\multicolumn{1}{l|}{Crime} & \multicolumn{1}{l|}{Finanças} & Infantil \\[5pt]

\multicolumn{1}{l|}{Negócio} & 
\multicolumn{1}{l|}{Documentário} & \multicolumn{1}{l|}{Cozinha} & Classica \\[5pt]

\multicolumn{1}{l|}{Tempo} & 
\multicolumn{1}{l|}{Drama} & \multicolumn{1}{l|}{Educação} & Country \\[5pt]

\multicolumn{1}{l|}{Comunicação} & 
\multicolumn{1}{l|}{Família} & \multicolumn{1}{l|}{Ficção} & Eletrônica \\[5pt]

\multicolumn{1}{l|}{Namoro} & 
\multicolumn{1}{l|}{Terror} & \multicolumn{1}{l|}{História} & Folk \\[5pt]

\multicolumn{1}{l|}{Educação} & 
\multicolumn{1}{l|}{Mistério} & \multicolumn{1}{l|}{Humor} & Rap \\[5pt]

\multicolumn{1}{l|}{Entretenimento} & 
\multicolumn{1}{l|}{Suspense} & \multicolumn{1}{l|}{Psicologia} & Jazz \\[5pt]

\multicolumn{1}{l|}{Eventos} & 
\multicolumn{1}{l|}{Ficção Científica} & \multicolumn{1}{l|}{Religião} & Metal \\[5pt]

\multicolumn{1}{l|}{Família} & 
\multicolumn{1}{l|}{Fantasia} & \multicolumn{1}{l|}{Romance} & Pop \\[5pt]

\multicolumn{1}{l|}{Finanças} & 
\multicolumn{1}{l|}{Esportes} & \multicolumn{1}{l|}{Fantasia} & Reggae \\[5pt]

\multicolumn{1}{l|}{Social} & 
\multicolumn{1}{l|}{} & \multicolumn{1}{l|}{Auto-ajuda} & Rock \\[5pt]

\multicolumn{1}{l|}{Viagens} & 
\multicolumn{1}{l|}{} & \multicolumn{1}{l|}{} & Sertanejo \\[5pt]

\multicolumn{1}{l|}{Saúde e Ginástica} & 
\multicolumn{1}{l|}{} & \multicolumn{1}{l|}{} & Samba \\[5pt]

\multicolumn{1}{l|}{Mapas e Navegação} & 
\multicolumn{1}{l|}{} & \multicolumn{1}{l|}{} & Rock Nacional \\[5pt]

\multicolumn{1}{l|}{Fotografia} & 
\multicolumn{1}{l|}{} & \multicolumn{1}{l|}{} & Indie \\ \bottomrule
\end{tabular}%
}
\caption{Categorias (com seus respectivos tipos) possíveis no sistema}
\end{table}


%% ===================================
\newpage
\section*{Modelo Conceitual}
\begin{figure}[H]
\includegraphics[angle=90, height=0.75\paperheight]{modeloConceitual.png}
\end{figure}

%% ===================================
\newpage
\section*{Dicionário de Dados}
O dicionário de dados será itemizado abaixo, com cada tabela representando uma entidade (ou relacionamento com atributos) presente na modelagem coneitual. \\
Atributos em negrito definem que eles representam chaves candidatas das entidades.Atributos com um asterisco indicam que eles são opcionais. Todos os tipos descritos abaixos são referentes aos tipos que o PostgreSQL implementa em seu SGBD.

\vspace{1cm}

\begin{table}[h]
\centering
\resizebox{\textwidth}{!}{%
\begin{tabular}{@{}llll@{}}
\toprule
\multicolumn{4}{c}{\textbf{Item}} \\ \midrule
\multicolumn{1}{l|}{\textit{Atributo}} & \multicolumn{1}{l|}{\textit{Tipo}} &
\multicolumn{1}{l|}{\textit{Descrição}} & \textit{Exemplo} \\ \midrule
\multicolumn{1}{l|}{\textbf{id}} & \multicolumn{1}{l|}{uuid} & \multicolumn{1}{l|}{\begin{tabular}[c]{@{}l@{}}Identificador único para o item\end{tabular}} & 704ef290-... \\[10pt]
\multicolumn{1}{l|}{price*} & \multicolumn{1}{l|}{numeric(5,2)} & \multicolumn{1}{l|}{\begin{tabular}[c]{@{}l@{}}Valor a ser cobrado por um item, em reais,\\ sendo o valor limitado em R\$ 999,99.\\ Caso esse campo esteja nulo, entendemos que\\ o produto está de graça.\end{tabular}} & 9.99 \\[25pt]
\multicolumn{1}{l|}{name} & \multicolumn{1}{l|}{varchar(50)} & \multicolumn{1}{l|}{\begin{tabular}[c]{@{}l@{}}Nome do item, com no máximo\\ 50 caracteres\end{tabular}} & Super SGBD \\[15pt]
\multicolumn{1}{l|}{release\_date} & \multicolumn{1}{l|}{date} & \multicolumn{1}{l|}{Data de lançamento do item} & 31/12/2018 \\[10pt]
\multicolumn{1}{l|}{about} & \multicolumn{1}{l|}{varchar(1000)} & \multicolumn{1}{l|}{\begin{tabular}[c]{@{}l@{}}Descrição sobre o item, podendo conter\\ até no máximo 1000 caracteres\end{tabular}} & Lorem ipsum... \\[15pt]
\multicolumn{1}{l|}{promotion\_date*} & \multicolumn{1}{l|}{date} & \multicolumn{1}{l|}{\begin{tabular}[c]{@{}l@{}}Informa se o item está ou não em promoção.\\ Se campo for diferente de null,\\e possuir uma data nele, sabemos que o item\\ estará em promoção ATÉ aquela data. \\Se a data atual for maior do que a\\ descrita no campo, ou ele for NULL,\\ então o produto NÃO está em promoção.\end{tabular}} & 19/07/2019 \\[50pt]
\multicolumn{1}{l|}{promotion\_price*} & \multicolumn{1}{l|}{numeric(5,2)} & \multicolumn{1}{l|}{\begin{tabular}[c]{@{}l@{}}Valor a ser cobrado por um item, em reais,\\ caso o atributo is\_promotion seja true,\\ sendo o valor limitado em R\$ 999,99.\\ Caso esse campo esteja nulo, enquanto\\ is\_promotion esteja em true,\\ entendemos que o produto está de graça.\end{tabular}} &  \\[40pt]
\multicolumn{1}{l|}{type} & \multicolumn{1}{l|}{char(5)} & \multicolumn{1}{l|}{\begin{tabular}[c]{@{}l@{}}Referencia qual tipo de Item esse Item realmente é. \\ É utilizado para sabermos em qual outra tabela da\\ especificação se encontram o resto dos\\ dados (já que a generalização/especificação\\ da tabela Item é total). Pode assumir os\\ valores \{ app, movie, book, album\}\end{tabular}} & app \\\bottomrule
\end{tabular}%
}
\caption{Tabela Item e a descrição textual de seus atributos}
\end{table}

\begin{table}[]
\centering
\resizebox{\textwidth}{!}{%
\begin{tabular}{@{}llll@{}}
\toprule
\multicolumn{4}{c}{\textbf{Attachment}} \\ \midrule
\multicolumn{1}{l|}{\textit{Atributo}} & \multicolumn{1}{l|}{\textit{Tipo}} &
\multicolumn{1}{l|}{\textit{Descrição}} & \textit{Exemplo} \\ \midrule
\multicolumn{1}{l|}{\textbf{filepath}} & \multicolumn{1}{l|}{varchar(100)} & \multicolumn{1}{l|}{\begin{tabular}[c]{@{}l@{}}Caminho para onde o anexo\\ está guardado localmente\end{tabular}} & ./downloads/attach\_001.jpg \\[15pt]
\multicolumn{1}{l|}{type\_attachment} & \multicolumn{1}{l|}{char(6)} & \multicolumn{1}{l|}{\begin{tabular}[c]{@{}l@{}}Tipo de anexo, sendo que os\\ possíveis valores são \{imagem, video\} \end{tabular}} & imagem \\[15pt]
\multicolumn{1}{l|}{name} & \multicolumn{1}{l|}{varchar(50)} & \multicolumn{1}{l|}{\begin{tabular}[c]{@{}l@{}}Nome dado ao anexo, para ser\\ utilizado na hora de mostrarmos\\ o mesmo na UI. \end{tabular}} & Imagem 01 \\[20pt]
\multicolumn{1}{l|}{extension} & \multicolumn{1}{l|}{char(5)} & \multicolumn{1}{l|}{\begin{tabular}[c]{@{}l@{}}Extensão dada para o arquivo \end{tabular}} & jpg \\[5pt]
\multicolumn{1}{l|}{app} & \multicolumn{1}{l|}{uuid} & \multicolumn{1}{l|}{\begin{tabular}[c]{@{}l@{}}Referência à tabela Item, \\representando o relacionamento Attaching.\\Cada um dos Attachments pertence a\\ somente um Item.\end{tabular}} & 704ef290-... \\\bottomrule
\end{tabular}%
}
\caption{Tabela Attachment e a descrição textual de seus atributos}
\end{table}

\begin{table}[]
\centering
\resizebox{\textwidth}{!}{%
\begin{tabular}{@{}llll@{}}
\toprule
\multicolumn{4}{c}{\textbf{User}} \\ \midrule
\multicolumn{1}{l|}{\textit{Atributo}} & \multicolumn{1}{l|}{\textit{Tipo}} &
\multicolumn{1}{l|}{\textit{Descrição}} & \textit{Exemplo} \\ \midrule
\multicolumn{1}{l|}{\textbf{email}} & \multicolumn{1}{l|}{varchar(50)} & \multicolumn{1}{l|}{\begin{tabular}[c]{@{}l@{}}E-mail do usuário, válidado de\\ acordo com a RFC 5322\cite{RFC5322}\end{tabular}} & example@example.com \\[15pt]
\multicolumn{1}{l|}{name} & \multicolumn{1}{l|}{varchar(80)} & \multicolumn{1}{l|}{\begin{tabular}[c]{@{}l@{}}Nome do usuário, com no\\ máximo 80 caracteres\end{tabular}} & John Doe \\[15pt]
\multicolumn{1}{l|}{password} & \multicolumn{1}{l|}{varchar} & \multicolumn{1}{l|}{\begin{tabular}[c]{@{}l@{}}Senha do usuário, guardada com\\ criptografia SHA512\cite{SHA512}\end{tabular}} & 1253AB... \\[15pt]
\multicolumn{1}{l|}{cpf} & \multicolumn{1}{l|}{char(11)} & \multicolumn{1}{l|}{\begin{tabular}[c]{@{}l@{}}CPF do usuário, armazenado\\ sem os caracteres separadores\end{tabular}} & 99999999999 \\[15pt]
\multicolumn{1}{l|}{birthdate} & \multicolumn{1}{l|}{date} & \multicolumn{1}{l|}{\begin{tabular}[c]{@{}l@{}}Data de nascimento do usuário\end{tabular}} & 02/01/1998 \\ \bottomrule
\end{tabular}%
}
\caption{Tabela User e a descrição textual de seus atributos}
\end{table}

\begin{table}[]
\centering
\resizebox{\textwidth}{!}{%
\begin{tabular}{@{}llll@{}}
\toprule
\multicolumn{4}{c}{\textbf{Credit Card}} \\ \midrule
\multicolumn{1}{l|}{\textit{Atributo}} & \multicolumn{1}{l|}{\textit{Tipo}} &
\multicolumn{1}{l|}{\textit{Descrição}} & \textit{Exemplo} \\ \midrule
\multicolumn{1}{l|}{\textbf{card\_number}} & \multicolumn{1}{l|}{char(16)} & \multicolumn{1}{l|}{\begin{tabular}[c]{@{}l@{}}Identificador do cartão de crédito\end{tabular}} & 1111 1111 1111 1111\\[5pt]
\multicolumn{1}{l|}{card\_name} & \multicolumn{1}{l|}{varchar(80)} & \multicolumn{1}{l|}{\begin{tabular}[c]{@{}l@{}}Nome do dono do cartão, que pode \\ser diferente do User relacionado a\\ esse Credit Card\end{tabular}} & John Doe\\[20pt]
\multicolumn{1}{l|}{due\_date} & \multicolumn{1}{l|}{date} & \multicolumn{1}{l|}{\begin{tabular}[c]{@{}l@{}}Data de vencimento do cartão\end{tabular}} & 31/12/2025\\[5pt]
\multicolumn{1}{l|}{\textbf{user}} & \multicolumn{1}{l|}{varchar(50)} & \multicolumn{1}{l|}{\begin{tabular}[c]{@{}l@{}}Referência à tabela User,\\ representando o relacionamento\\ PaymentMethod. Cada CreditCard\\ pertence a apenas um usuário.\end{tabular}} & example@example.com\\\bottomrule
\end{tabular}%
}
\caption{Tabela CreditCard e a descrição textual de seus atributos}
\end{table}

\begin{table}[]
\centering
\resizebox{\textwidth}{!}{%
\begin{tabular}{@{}llll@{}}
\toprule
\multicolumn{4}{c}{\textbf{WishList}} \\ \midrule
\multicolumn{1}{l|}{\textit{Atributo}} & \multicolumn{1}{l|}{\textit{Tipo}} &
\multicolumn{1}{l|}{\textit{Descrição}} & \textit{Exemplo} \\ \midrule
\multicolumn{1}{l|}{\textbf{user}} & \multicolumn{1}{l|}{varchar(50)} & \multicolumn{1}{l|}{\begin{tabular}[c]{@{}l@{}}Referência à tabela User,\\ representando o relacionamento dele\\ com o Relacionamento WishList\\ (representado por essa tabela). Cada\\ entrada na WishList pertence a um\\ e somente um usuário.\end{tabular}} & example@example.com \\[40pt]
\multicolumn{1}{l|}{\textbf{app}} & \multicolumn{1}{l|}{uuid} & \multicolumn{1}{l|}{\begin{tabular}[c]{@{}l@{}}Referência à tabela Item,\\ representando o relacionamento dele\\ com o Relacionamento WishList\\ (representado por essa tabela). Cada\\ entrada na WishList pertence a um\\ e somente um item.\end{tabular}} & 704ef290-... \\[40pt]
\multicolumn{1}{l|}{date} & \multicolumn{1}{l|}{date} & \multicolumn{1}{l|}{\begin{tabular}[c]{@{}l@{}}Data na qual o Item foi adicionado\\ à WishList do User\end{tabular}} & 12/01/2019 \\ \bottomrule
\end{tabular}%
}
\caption{Tabela WishList e a descrição textual de seus atributos}
\end{table}

\begin{table}[]
\centering
\resizebox{\textwidth}{!}{%
\begin{tabular}{@{}llll@{}}
\toprule
\multicolumn{4}{c}{\textbf{Review}} \\ \midrule
\multicolumn{1}{l|}{\textit{Atributo}} & \multicolumn{1}{l|}{\textit{Tipo}} &
\multicolumn{1}{l|}{\textit{Descrição}} & \textit{Exemplo} \\ \midrule
\multicolumn{1}{l|}{\textbf{user}} & \multicolumn{1}{l|}{varchar(50)} &
\multicolumn{1}{l|}{\begin{tabular}[c]{@{}l@{}}Referência à tabela User,\\ representando o relacionamento dele\\ com o Relacionamento Review\\ (representado por essa tabela). Cada\\ entrada em Review pertence a um\\ e somente um usuário. \end{tabular}} & example@example.com \\[40pt]
\multicolumn{1}{l|}{\textbf{item}} & \multicolumn{1}{l|}{uuid} & \multicolumn{1}{l|}{\begin{tabular}[c]{@{}l@{}}Referência à tabela Item,\\ representando o relacionamento dele\\ com o Relacionamento Review\\ (representado por essa tabela). Cada\\ entrada em Review pertence a um\\ e somente um item.\end{tabular}} & 704ef290-... \\[40pt]
\multicolumn{1}{l|}{rating} & \multicolumn{1}{l|}{smallint} & \multicolumn{1}{l|}{\begin{tabular}[c]{@{}l@{}}Nota atribuída pelo User ao Item.\\ Seus valores possíveis são \{0, 1, 2, 3, 4, 5\}\end{tabular}} & 4 \\[15pt]
\multicolumn{1}{l|}{date} & \multicolumn{1}{l|}{date} & \multicolumn{1}{l|}{\begin{tabular}[c]{@{}l@{}}Data que a review aconteceu\end{tabular}} & 22/02/2019 \\[10pt]
\multicolumn{1}{l|}{comment*} & \multicolumn{1}{l|}{varchar(256)} & \multicolumn{1}{l|}{\begin{tabular}[c]{@{}l@{}}Comentário opcional do User referente ao Item\end{tabular}} & Top \\ \bottomrule
\end{tabular}%
}
\caption{Tabela Review e a descrição textual de seus atributos}
\end{table}

\begin{table}[]
\centering
\resizebox{\textwidth}{!}{%
\begin{tabular}{@{}llll@{}}
\toprule
\multicolumn{4}{c}{\textbf{Download}} \\ \midrule
\multicolumn{1}{l|}{\textit{Atributo}} & \multicolumn{1}{l|}{\textit{Tipo}} &
\multicolumn{1}{l|}{\textit{Descrição}} & \textit{Exemplo} \\ \midrule
\multicolumn{1}{l|}{\textbf{item}} & \multicolumn{1}{l|}{uuid} & \multicolumn{1}{l|}{\begin{tabular}[c]{@{}l@{}}Referência à tabela Item,\\ representando o relacionamento dele\\ com a Entidade Associativa Download\\ (representado por essa tabela). Cada\\ entrada em download pertence a um\\ e somente um item.\end{tabular}} & 704ef290-... \\[40pt]
\multicolumn{1}{l|}{\textbf{user}} & \multicolumn{1}{l|}{varchar(50)} & \multicolumn{1}{l|}{\begin{tabular}[c]{@{}l@{}}Referência à tabela User,\\ representando o relacionamento dele\\ com a Entidade Associativa Download\\ (representado por essa tabela). Cada\\ entrada em download pertence a um\\ e somente um user.\end{tabular}} & example@example.com \\[40pt]
\multicolumn{1}{l|}{credit\_card*} & \multicolumn{1}{l|}{char(16)} & \multicolumn{1}{l|}{\begin{tabular}[c]{@{}l@{}}Referência à tabela CreditCard,\\ representando o relacionamento Purchase,\\ representando qual CreditCard\\ o User utilizou para realizar esse download.\\ Esse campo só está preenchido se quando o\\ User realizou o download do Item,\\ ele possuía price ou promotion\_price\\ (caso em promoção) diferente de null ou 0.\end{tabular}} & 1111 1111 1111 1111 \\[50pt]
\multicolumn{1}{l|}{first\_download\_date} & \multicolumn{1}{l|}{date} & \multicolumn{1}{l|}{\begin{tabular}[c]{@{}l@{}}Data na qual o item foi baixado pela\\ primeira vez\end{tabular}} & 18/01/2019 \\[15pt]
\multicolumn{1}{l|}{last\_download\_date} & \multicolumn{1}{l|}{date} & \multicolumn{1}{l|}{\begin{tabular}[c]{@{}l@{}}Data na qual o item foi baixado pela\\ última vez. É sempre maior ou igual a\\ first\_download\_date.\end{tabular}} & 18/01/2019 \\[20pt]
\multicolumn{1}{l|}{downloaded} & \multicolumn{1}{l|}{boolean} & \multicolumn{1}{l|}{\begin{tabular}[c]{@{}l@{}}Informa se o item está instalado (true)\\ ou desinstalado (false) no momento\end{tabular}} & false \\ \bottomrule
\end{tabular}%
}\caption{Tabela Download e a descrição textual de seus atributos}
\end{table}

\begin{table}[]
\centering
\resizebox{\textwidth}{!}{%
\begin{tabular}{@{}llll@{}}
\toprule
\multicolumn{4}{c}{\textbf{Language}} \\ \midrule
\multicolumn{1}{l|}{\textit{Atributo}} & \multicolumn{1}{l|}{\textit{Tipo}} &
\multicolumn{1}{l|}{\textit{Descrição}} & \textit{Exemplo} \\ \midrule
\multicolumn{1}{l|}{\textbf{name}} & \multicolumn{1}{l|}{varchar(20)} & \multicolumn{1}{l|}{\begin{tabular}[c]{@{}l@{}}Nome da linguagem, com no máximo\\ 20 caracteres\end{tabular}} & italiano \\ \bottomrule
\end{tabular}%
}
\caption{Tabela Language e a descrição textual de seus atributos}
\end{table}

\begin{table}[]
\centering
\resizebox{\textwidth}{!}{%
\begin{tabular}{@{}llll@{}}
\toprule
\multicolumn{4}{c}{\textbf{Developer}} \\ \midrule
\multicolumn{1}{l|}{\textit{Atributo}} & \multicolumn{1}{l|}{\textit{Tipo}} &
\multicolumn{1}{l|}{\textit{Descrição}} & \textit{Exemplo} \\ \midrule
\multicolumn{1}{l|}{\textbf{email}} & \multicolumn{1}{l|}{varchar(50)} & \multicolumn{1}{l|}{\begin{tabular}[c]{@{}l@{}}E-mail do(a) desenvolvedor(a),\\ validado de acordo com a RFC 5322\cite{RFC5322}\end{tabular}} & developer@example.com \\[15pt]
\multicolumn{1}{l|}{name} & \multicolumn{1}{l|}{varchar(80)} & \multicolumn{1}{l|}{\begin{tabular}[c]{@{}l@{}}Nome do(a) desenvolvedor(a),\\ com no máximo 80 caracteres\end{tabular}} & Developer Doe \\[15pt]
\multicolumn{1}{l|}{address} & \multicolumn{1}{l|}{varchar} & \multicolumn{1}{l|}{\begin{tabular}[c]{@{}l@{}}Endereço do(a) desenvolvedor(a)\\ contendo rua, número, bairro, cidade\\ e país, ou conforme for o padrão do país\\ de origem do(a) desenvolvedor(a) \end{tabular}} & ..., Havana, Cuba \\[25pt]
\multicolumn{1}{l|}{password} & \multicolumn{1}{l|}{varchar} & \multicolumn{1}{l|}{\begin{tabular}[c]{@{}l@{}}Senha do developer, guardada com\\ criptografia SHA512\cite{SHA512} \end{tabular}} & 8269bf0a... \\\bottomrule
\end{tabular}%
}
\caption{Tabela Developer e a descrição textual de seus atributos}
\end{table}

\begin{table}[]
\centering
\resizebox{\textwidth}{!}{%
\begin{tabular}{@{}llll@{}}
\toprule
\multicolumn{4}{c}{\textbf{Person}} \\ \midrule
\multicolumn{1}{l|}{\textit{Atributo}} & \multicolumn{1}{l|}{\textit{Tipo}} &
\multicolumn{1}{l|}{\textit{Descrição}} & \textit{Exemplo} \\ \midrule
\multicolumn{1}{l|}{\textbf{name}} & \multicolumn{1}{l|}{varchar(80)} & \multicolumn{1}{l|}{\begin{tabular}[c]{@{}l@{}}Representa o nome de uma pessoa\\ que pode ter escrito algum Book,\\ participado de algum Movie\\ ou sido a criadora de algum Album.\\ Nome pode possuir no máximo 80 caracteres\end{tabular}} & Person Doe \\ \bottomrule
\end{tabular}%
}
\caption{Tabela Artist e a descrição textual de seus atributos}
\end{table}


\begin{table}[]
\centering
\resizebox{\textwidth}{!}{%
\begin{tabular}{@{}llll@{}}
\toprule
\multicolumn{4}{c}{\textbf{Category}} \\ \midrule
\multicolumn{1}{l|}{\textit{Atributo}} & \multicolumn{1}{l|}{\textit{Tipo}} &
\multicolumn{1}{l|}{\textit{Descrição}} & \textit{Exemplo} \\ \midrule
\multicolumn{1}{l|}{\textbf{name}} & \multicolumn{1}{l|}{varchar(25)} & \multicolumn{1}{l|}{\begin{tabular}[c]{@{}l@{}}Nome da categoria, sendo único\\ dentro do mesmo item\_type\end{tabular}} & Ficção Científica \\[15pt]
\multicolumn{1}{l|}{\textbf{item\_type}} & \multicolumn{1}{l|}{char(5)} & \multicolumn{1}{l|}{\begin{tabular}[c]{@{}l@{}}Tipo de item à qual essa categoria\\ se aplica. Os valores possíveis são\\ \{app, livro, filme, album\}. Para o\\ mesmo item\_type, não podem haver 2\\ categorias com o mesmo nome. \\ Isso significa que item\_type e name \\ \textbf{juntos}, identificam uma categoria.\end{tabular}} & filme \\ \bottomrule
\end{tabular}%
}
\caption{Tabela Category e a descrição textual de seus atributos. Mais informações sobre ela na Tabela 2.}
\end{table}

\begin{table}[]
\centering
\resizebox{\textwidth}{!}{%
\begin{tabular}{@{}llll@{}}
\toprule
\multicolumn{4}{c}{\textbf{Categorization}} \\ \midrule
\multicolumn{1}{l|}{\textit{Atributo}} & \multicolumn{1}{l|}{\textit{Tipo}} &
\multicolumn{1}{l|}{\textit{Descrição}} & \textit{Exemplo} \\ \midrule
\multicolumn{1}{l|}{\textbf{category\_name}} & \multicolumn{1}{l|}{varchar(25)} & \multicolumn{1}{l|}{\begin{tabular}[c]{@{}l@{}}Referência à tabela Category,\\ representando o relacionamento dele\\ com o Relacionamento Categorization\\ (representado por essa tabela). Cada\\ entrada em Categorization pertence a um\\ e somente uma categoria,\\ que possui esse nome e o tipo descrito\\ no campo abaixo.\end{tabular}} & Ficção Científica \\[55pt]
\multicolumn{1}{l|}{\textbf{category\_item\_type}} & \multicolumn{1}{l|}{char(5)} & \multicolumn{1}{l|}{\begin{tabular}[c]{@{}l@{}}Referência à tabela Category,\\ representando o relacionamento dele\\ com o Relacionamento Categorization\\ (representado por essa tabela). Cada\\ entrada em Categorization pertence a um\\ e somente uma categoria,\\ que possui o nome descrito acima e esse tipo.\\ O Item referenciado por essa Categorization\\ precisa ser do mesmo tipo que o descrito\\ nesse campo.\end{tabular}} & filme \\[65pt]
\multicolumn{1}{l|}{\textbf{item}} & \multicolumn{1}{l|}{uuid} & \multicolumn{1}{l|}{\begin{tabular}[c]{@{}l@{}}Referência à tabela Item,\\ representando o relacionamento dele\\ com o Relacionamento Categorization\\ (representado por essa tabela). Cada\\ entrada em Categorization pertence a um\\ e somente um Item.\end{tabular}} & 704ef290-... \\ \bottomrule
\end{tabular}%
}
\caption{Tabela Categorization e a descrição textual de seus atributos}
\end{table}

\begin{table}[]
\centering
\resizebox{\textwidth}{!}{%
\begin{tabular}{@{}llll@{}}
\toprule
\multicolumn{4}{c}{\textbf{App \textit{(que especifica a tabela Item)}}} \\ \midrule
\multicolumn{1}{l|}{\textit{Atributo}} & \multicolumn{1}{l|}{\textit{Tipo}} &
\multicolumn{1}{l|}{\textit{Descrição}} & \textit{Exemplo} \\ \midrule
\multicolumn{1}{l|}{\textbf{item}} & \multicolumn{1}{l|}{uuid} & \multicolumn{1}{l|}{\begin{tabular}[c]{@{}l@{}}Referência à tabela Item, representando o\\ relacionamento de especialização com Item.\\ Cada App pertence a apenas um item.\\ Esse relacionamento é utilizado para termos acesso\\ às outras informações de App.\end{tabular}} & 704ef290-... \\[45pt]
\multicolumn{1}{l|}{developer} & \multicolumn{1}{l|}{varchar(50)} & \multicolumn{1}{l|}{\begin{tabular}[c]{@{}l@{}}Referência à tabela Developer,\\ representando o relacionamento\\ Development. Cada App\\ possui apenas um developer.\end{tabular}} & developer@example.com \\[20pt]
\multicolumn{1}{l|}{size} & \multicolumn{1}{l|}{bigint} & \multicolumn{1}{l|}{\begin{tabular}[c]{@{}l@{}}Tamanho do aplicativo, em bytes,\\ limitado a 1TB (1099511627800)\end{tabular}} & 1048576 \\[15pt]
\multicolumn{1}{l|}{version} & \multicolumn{1}{l|}{varchar} & \multicolumn{1}{l|}{\begin{tabular}[c]{@{}l@{}}Texto demonstrando a versão do app no formato\\ "MAJOR.MINOR.PATCH", sendo que MAJOR, MINOR\\ e PATCH são números inteiros, sem vírgulas, e PATCH\\ pode ser omitido juntamente com o ponto anterior à ele \end{tabular}} & 2.1.12 \\ \bottomrule
\end{tabular}%
}
\caption{Tabela App (que especifica a tabela Item) e a descrição textual de seus atributos}
\end{table}

\begin{table}[]
\centering
\resizebox{\textwidth}{!}{%
\begin{tabular}{@{}llll@{}}
\toprule
\multicolumn{4}{c}{\textbf{Album \textit{(que especifica a tabela Item)}}} \\ \midrule
\multicolumn{1}{l|}{\textit{Atributo}} & \multicolumn{1}{l|}{\textit{Tipo}} &
\multicolumn{1}{l|}{\textit{Descrição}} & \textit{Exemplo} \\ \midrule
\multicolumn{1}{l|}{\textbf{item}} & \multicolumn{1}{l|}{uuid} & \multicolumn{1}{l|}{\begin{tabular}[c]{@{}l@{}}Referência à tabela Item, representando o\\ relacionamento de especialização com Item.\\ Cada Album pertence a apenas um item.\\ Esse relacionamento é utilizado para termos acesso\\ às outras informações de Album.\end{tabular}} & 704ef290-... \\[20pt]
\multicolumn{1}{l|}{\textbf{artist}} & \multicolumn{1}{l|}{varchar(80)} & \multicolumn{1}{l|}{\begin{tabular}[c]{@{}l@{}}Referência à tabela Person,\\ representando o relacionamento\\ Album\_Authorship. Cada Album\\ pertence a apenas um criador de álbum.\end{tabular}} & Artist Doe \\ \bottomrule
\end{tabular}%
}
\caption{Tabela Album (que especifica a tabela Item) e a descrição textual de seus atributos}
\end{table}

\begin{table}[]
\centering
\resizebox{\textwidth}{!}{%
\begin{tabular}{@{}llll@{}}
\toprule
\multicolumn{4}{c}{\textbf{Track}} \\ \midrule
\multicolumn{1}{l|}{\textit{Atributo}} & \multicolumn{1}{l|}{\textit{Tipo}} &
\multicolumn{1}{l|}{\textit{Descrição}} & \textit{Exemplo} \\ \midrule
\multicolumn{1}{l|}{\textbf{item}} & \multicolumn{1}{l|}{uuid} & \multicolumn{1}{l|}{\begin{tabular}[c]{@{}l@{}}Referência à tabela Album,\\ representando o relacionamento\\ Album\_Track. Cada Track\\ pertence a apenas um Album.\\ O artista que criou essa música pode\\ ser acessado através do álbum.\end{tabular}} & 704ef290-... \\[20pt]
\multicolumn{1}{l|}{\textbf{name}} & \multicolumn{1}{l|}{varchar(50)} & \multicolumn{1}{l|}{\begin{tabular}[c]{@{}l@{}}Nome da música, sendo única dentro de um álbum.\\ Possui no máximo 50 caracteres\end{tabular}} & Juntos e Shallow Now \\[15pt]
\multicolumn{1}{l|}{duration} & \multicolumn{1}{l|}{int} & \multicolumn{1}{l|}{\begin{tabular}[c]{@{}l@{}}Duração da música em segundos\end{tabular}} & 205 \\\bottomrule
\end{tabular}%
}
\caption{Tabela Track e a descrição textual de seus atributos}
\end{table}

\begin{table}[]
\centering
\resizebox{\textwidth}{!}{%
\begin{tabular}{@{}llll@{}}
\toprule
\multicolumn{4}{c}{\textbf{Book \textit{(que especifica a tabela Item)}}} \\ \midrule
\multicolumn{1}{l|}{\textit{Atributo}} & \multicolumn{1}{l|}{\textit{Tipo}} &
\multicolumn{1}{l|}{\textit{Descrição}} & \textit{Exemplo} \\ \midrule
\multicolumn{1}{l|}{\textbf{item}} & \multicolumn{1}{l|}{uuid} & \multicolumn{1}{l|}{\begin{tabular}[c]{@{}l@{}}Referência à tabela Item, representando o\\ relacionamento de especialização com Item.\\ Cada Book pertence a apenas um item.\\ Esse relacionamento é utilizado para termos acesso\\ às outras informações de Book.\end{tabular}} & 704ef290-... \\[30pt]
\multicolumn{1}{l|}{author} & \multicolumn{1}{l|}{varchar(80)} & \multicolumn{1}{l|}{\begin{tabular}[c]{@{}l@{}}Referência à tabela Person,\\ representando o relacionamento\\ Book\_Authorship. Cada Book\\ pertence a apenas um escritor.\end{tabular}} & Author Doe \\[25pt]
\multicolumn{1}{l|}{language} & \multicolumn{1}{l|}{varchar(80)} & \multicolumn{1}{l|}{\begin{tabular}[c]{@{}l@{}}Referência à tabela Language,\\ representando o relacionamento\\ Book\_Language. Cada Livro\\ possui apenas uma linguagem.\\ O mesmo livro, em diferentes linguages,\\ são representados como diferentes livros.\end{tabular}} & Italiano \\[40pt]
\multicolumn{1}{l|}{number\_of\_pages} & \multicolumn{1}{l|}{smallint} & \multicolumn{1}{l|}{\begin{tabular}[c]{@{}l@{}}Quantidade de páginas no livro\end{tabular}} & 425 \\[5pt]
\multicolumn{1}{l|}{\underline{ISBN}} & \multicolumn{1}{l|}{char(13)} & \multicolumn{1}{l|}{\begin{tabular}[c]{@{}l@{}}ISBN (código identificador) do livro\cite{ISBN}\end{tabular}} & 9788533302273 \\ \bottomrule
\end{tabular}%
}
\caption{Tabela Book (que especifica a tabela Item) e a descrição textual de seus atributos}
\end{table}


\begin{table}[]
\centering
\resizebox{\textwidth}{!}{%
\begin{tabular}{@{}llll@{}}
\toprule
\multicolumn{4}{c}{\textbf{Movie \textit{(que especifica a tabela Item)}}} \\ \midrule
\multicolumn{1}{l|}{\textit{Atributo}} & \multicolumn{1}{l|}{\textit{Tipo}} &
\multicolumn{1}{l|}{\textit{Descrição}} & \textit{Exemplo} \\ \midrule
\multicolumn{1}{l|}{\textbf{item}} & \multicolumn{1}{l|}{uuid} & \multicolumn{1}{l|}{\begin{tabular}[c]{@{}l@{}}Referência à tabela Item, representando o\\ relacionamento de especialização com Item.\\ Cada Movie pertence a apenas um item.\\ Esse relacionamento é utilizado para termos acesso\\ às outras informações de Movie.\end{tabular}} & 704ef290-... \\[20pt]
\multicolumn{1}{l|}{duration} & \multicolumn{1}{l|}{int} & \multicolumn{1}{l|}{\begin{tabular}[c]{@{}l@{}}Quantidade de minutos do filme\end{tabular}} & 123 \\ \bottomrule
\end{tabular}%
}
\caption{Tabela Movie (que especifica a tabela Item) e a descrição textual de seus atributos}
\end{table}

\begin{table}[]
\centering
\resizebox{\textwidth}{!}{%
\begin{tabular}{@{}llll@{}}
\toprule
\multicolumn{4}{c}{\textbf{Movie\_Language}} \\ \midrule
\multicolumn{1}{l|}{\textit{Atributo}} & \multicolumn{1}{l|}{\textit{Tipo}} &
\multicolumn{1}{l|}{\textit{Descrição}} & \textit{Exemplo} \\ \midrule
\multicolumn{1}{l|}{\textbf{movie}} & \multicolumn{1}{l|}{uuid} & \multicolumn{1}{l|}{\begin{tabular}[c]{@{}l@{}}Referência à tabela Movie,\\ representando o relacionamento dele\\ com o Relacionamento Movie\_Language\\ (representado por essa tabela). Cada\\ entrada em Movie\_Language se refere a um\\ e somente um Movie.\\ Cada Movie pode ter várias linguages, desde\\ que não seja descrito duas vezes\\ a mesma linguagem com o mesmo tipo.\end{tabular}} & 704ef290-... \\[50pt]
\multicolumn{1}{l|}{\textbf{language\_name}} & \multicolumn{1}{l|}{varchar(80)} & \multicolumn{1}{l|}{\begin{tabular}[c]{@{}l@{}}Referência à tabela Language,\\ representando o relacionamento dele\\ com o Relacionamento Movie\_Language\\ (representado por essa tabela). Cada\\ entrada em Movie\_Language se refere a um\\ e somente uma Linguagem.\end{tabular}} & Italiano \\[20pt]
\multicolumn{1}{l|}{\textbf{type}} & \multicolumn{1}{l|}{char(7)} & \multicolumn{1}{l|}{\begin{tabular}[c]{@{}l@{}}Representa se essa linguagem para esse filme\\ é relacionada à legenda ou ao áudio.\\ Os únicos valores permitidos são \{audio, legenda\}\end{tabular}} & legenda \\ \bottomrule
\end{tabular}%
}
\caption{Tabela Movie\_Language e a descrição textual de seus atributos}
\end{table}

\begin{table}[]
\centering
\resizebox{\textwidth}{!}{%
\begin{tabular}{@{}llll@{}}
\toprule
\multicolumn{4}{c}{\textbf{Movie\_Cast}} \\ \midrule
\multicolumn{1}{l|}{\textit{Atributo}} & \multicolumn{1}{l|}{\textit{Tipo}} &
\multicolumn{1}{l|}{\textit{Descrição}} & \textit{Exemplo} \\ \midrule
\multicolumn{1}{l|}{\textbf{movie\_id}} & \multicolumn{1}{l|}{uuid} & \multicolumn{1}{l|}{\begin{tabular}[c]{@{}l@{}}Referência à tabela Movie,\\ representando o relacionamento dele\\ com o Relacionamento Movie\_Cast\ (representado por essa tabela). Cada\\ entrada em Movie\_Cast se refere a um\\ e somente um Movie.\end{tabular}} & 704ef290-... \\[20pt]
\multicolumn{1}{l|}{\textbf{cast\_name}} & \multicolumn{1}{l|}{varchar(80)} & \multicolumn{1}{l|}{\begin{tabular}[c]{@{}l@{}}Referência à tabela Person,\\ representando o relacionamento dele\\ com o Relacionamento Movie\_Language\\ (representado por essa tabela). Cada\\ entrada em Movie\_Language se refere a um\\ e somente uma Person.\\ Essas pessoas podem participar de filmes,\\ ter escrito algum livro ou\\ produzido algum álbum;\end{tabular}} & Cast Doe \\ \bottomrule
\end{tabular}%
}
\caption{Tabela Movie\_Cast e a descrição textual de seus atributos}
\end{table}

%% ===================================
\clearpage
\section*{Modelo Lógico}
\begin{figure}[H]
\includegraphics[angle=90, height=0.75\paperheight]{modeloRelacional.png}
\end{figure}

%% ===================================
\newpage
\section*{Descrição do Mapeamento (Conceitual para Lógico)}

Segue uma lista de mapeamentos implementados, onde os atributos, entidades e relacionamentos que estão escritos com  itálico, são campos adicionais que não estavam no modelo conceitual e os que são definidos como "Mapeamento direto" querem dizer que estão da mesma maneira como foram definidos no conceitual.

Entidades:
\begin{itemize}
    \item User: Mapeamento direto para {\textbf{User}}.
    \begin{itemize}
        \item name: Mapeamento direto para {\textbf{User.name}}, obrigatório.
        \item email: Mapeamento direto para {\textbf{User.email}}, obrigatório e chave primária.
        \item date\_of\_birth: Mapeamento direto para {\textbf{User.date\_of\_birth}}, obrigatório.
        \item password: Mapeamento direto para {\textbf{User.password}}, obrigatório.
    \end{itemize}
    \item CreditCard: Mapeamento direto para {\textbf{CreditCard}}.
    \begin{itemize}
        \item due\_date: Mapeamento direto para {\textbf{CreditCard.due\_date}}, obrigatório.
        \item card\_name: Mapeamento direto para {\textbf{CreditCard.card\_name}}, obrigatório.
        \item card\_number: Mapeamento direto para {\textbf{CreditCard.card\_number}}, obrigatório e chave primária.
        \item{\textit{user\_email}}: Identificador do usuário, por conta do mapeamento de relacionamento por colunas adicionais no relacionamento PaymentMethod. Mapeado para {\textbf{CreditCard.user\_email}}, obrigatório e chave estrangeira.
    \end{itemize}
    \item Item: Como é uma generalização total e exclusiva, o mapeamento foi feito seguindo a Alternativa 1 dos slides de mapeamento(implementação por diferentes relações com uma relação para dados comuns). Foi usada como chave estrangeira (única) na entidade especializada a chave primária da entidade generalizada, fazendo com que sempre que ocorre uma entidade especializada, exista uma entidade generalizada que ainda não foi relacionada. Mapeada para {\textbf{Item}}.
    \begin{itemize}
        \item {\textit{id}}: Valor adicional criado para ser a chave primária do item, pois não existe nenhuma chave candidate que, sozinha, seja determinante para o Item. Como esse valor terá que que ser usado como chave estrangeira na entidade especializada, foi decidido que seria lógico usar uma cháve primária não-composta, por conta do menor espaço ocupado e facilidade. Mapeado para  {\textbf{Item.id}}, obrigatório e chave primária, sendo do tipo UUID, automaticamente gerado pelo banco de dados.
        \item name: Mapeamento direto para {\textbf{Item.name}}, obrigatório.
        \item price: Mapeamento direto para {\textbf{Item.price}}, opcional.
        \item release\_date: Mapeamento direto para {\textbf{Item.release\_date}}, obrigatório.
        \item promotion\_date: Mapeamento direto para {\textbf{Item.promotion\_date}}, opcional.
        \item promotion\_price: Mapeamento direto para {\textbf{Item.promotion\_price}}, opcional.
        \item about: Mapeamento direto para {\textbf{Item.about}}, obrigatório.
    \end{itemize}
    \item App: Entidade especializada de item, onde sua chave primária é uma chave estrangeira para item, já que, como dito anteriormente, foi utilizada a Alternativa 1 dos slides de mapeamento de generalização. Mapeada para {\textbf{App}}.
    \begin{itemize}
        \item {\textit{item\_id}}: É uma chave estrangeira para Item, ao mesmo tempo em que é chave primária, fazendo com que seja único o relacionamento com Item e não exista nenhum outro App relacionado com o mesmo. Criado Mapeamento direto para {\textbf{App.item\_id}}, obrigatório e chave primária.
        \item {\textit{developer\_email}}: É uma chave estrangeira para Developer. Mapeado para {\textbf{App. developer\_email}}, obrigatório.
        \item size: Mapeamento direto para {\textbf{App.size}}, obrigatório.
        \item version: Mapeamento direto para {\textbf{App.version}}, obrigatório.
    \end{itemize}
    \item Book: Entidade especializada de item, onde sua chave primária é uma chave estrangeira para item, já que, como dito anteriormente, foi utilizada a Alternativa 1 dos slides de mapeamento de generalização. Mapeada para {\textbf{Book}}.
    \begin{itemize}
        \item {\textit{item\_id}}: É uma chave estrangeira para Item, ao mesmo tempo em que é chave primária, fazendo com que seja único o relacionamento com Item e não exista nenhum outro App relacionado com o mesmo. Mapeamento direto para {\textbf{Book.item\_id}}, obrigatório e chave primária.
        \item {\textit{author\_name}}: Chave estrangeira para a tabela Author. Criado por conta do mapeamento de relacionamento por colunas adicionais no relacionamento Book\_Authorship. Mapeado para {\textbf{Book. author\_name}}, obrigatório.
        \item {\textit{language\_name}}: Chave estrangeira para a tabela Language. Criado por conta do mapeamento de relacionamento por colunas adicionais no relacionamento Book\_Language. Mapeado para {\textbf{Book. language\_name}}, obrigatório.
        \item isbn: Mapeamento direto para {\textbf{Book.isbn}}, obrigatório e único.
        \item number\_of\_pages: Mapeamento direto para {\textbf{Book.number\_of\_pages}}, obrigatório.
    \end{itemize}
    \item Movie: Entidade especializada de item, onde sua chave primária é uma chave estrangeira para item, já que, como dito anteriormente, foi utilizada a Alternativa 1 dos slides de mapeamento de generalização. Mapeada para {\textbf{Movie}}.
    \begin{itemize}
        \item {\textit{item\_id}}: É uma chave estrangeira para Item, ao mesmo tempo em que é chave primária, fazendo com que seja único o relacionamento com Item e não exista nenhum outro App relacionado com o mesmo. Mapeamento direto para {\textbf{Movie.item\_id}}, obrigatório e chave primária.
        \item duraration: Mapeamento direto para {\textbf{Movie.duration}}, obrigatório.
    \end{itemize}
    \item Album: Entidade especializada de item, onde sua chave primária é uma chave estrangeira para item, já que, como dito anteriormente, foi utilizada a Alternativa 1 dos slides de mapeamento de generalização. Mapeada para {\textbf{Album}}.
    \begin{itemize}
        \item {\textit{item\_id}}: É uma chave estrangeira para Item, ao mesmo tempo em que é chave primária, fazendo com que seja único o relacionamento com Item e não exista nenhum outro App relacionado com o mesmo. Mapeamento direto para {\textbf{Album.item\_id}}, obrigatório e chave primária.
        \item {\textit{artist\_name}}: É uma chave estrangeira para Album. Criado por conta do mapeamento de relacionamento por colunas adicionais no relacionamento Album\_Authorship. Mapeado para {\textbf{Album. artist\_name}}, obrigatório.
    \end{itemize}
    \item Developer: Mapeamento direto para {\textbf{Developer}}.
    \begin{itemize}
        \item email: Mapeamento direto para {\textbf{Developer.email}}, obrigatório e chave primária, pois é a única chave candidata.
        \item name: Mapeamento direto para {\textbf{developer.name}}, obrigatório.
        \item address: Mapeamento direto para {\textbf{Developer.address}}, obrigatório.
        \item password: Mapeamento direto para {\textbf{Developer.password}}, obrigatório.
    \end{itemize}
    \item Author: Mapeamento direto para {\textbf{Author}}.
    \begin{itemize}
        \item name: Mapeamento direto para {\textbf{Author.name}}, obrigatório e chave primária, pois é a única chave candidata, além de ser o único atributo.
    \end{itemize}
    \item Language: Mapeamento direto para {\textbf{Language}}.
    \begin{itemize}
        \item name: Mapeamento direto para {\textbf{Language.name}}, obrigatório e chave primária, pois é a única chave candidat, além de ser o único atributoa.
    \end{itemize}
    \item Cast: Mapeamento direto para {\textbf{Cast}}.
    \begin{itemize}
        \item name: Mapeamento direto para {\textbf{Cast.name}}, obrigatório e chave primária, pois é a única chave candidata, além de ser o único atributo.
    \end{itemize}
    \item Track: Mapeamento direto para {\textbf{Track}}.
    \begin{itemize}
        \item name: Mapeamento direto para {\textbf{Track.name}}, obrigatório e chave primária juntamente com \textbf{Track.album\_id}, já que \textbf{Track} é uma entidade fraca.
        \item \textit{album\_id}: Mapeamento direto para {\textbf{Track.album\_id}}, obrigatório e chave primária juntamente com \textbf{Track.name}, já que \textbf{Track} é uma entidade fraca em relação a \textbf{Album}. Criado por conta do mapeamento de relacionamento por colunas adicionais no relacionamento Album\_Track.
        \item duration: Mapeamento direto para {\textbf{Track.duration}}, obrigatório e não nulo.
    \end{itemize}
    \item Attachment: Mapeamento direto para {\textbf{Attachment}}
    \begin{itemize}
        \item filepath: Mapeamento direto para \textbf{Attachment.filepath}, obrigatório e não nulo, é a chave primária da tabela por ser a única chave candidata.
        \item \textit{item\_id}: Mapeamento direto para \textbf{Attachment.item\_id}, obrigatório e não nulo. Criado por conta do mapeamento de relacionamento por colunas adicionais no relacionamento Attaching. É uma chave estrangeira para a tabela Item.
        \item name: Mapeamento direto para \textbf{Attachment.name}, obrigatório e não nulo.
        \item type\_attachment: Mapeamento direto para \textbf{Attachment.type\_attachment}, obrigatório e não nulo.
        \item extension: Mapeamento direto para \textbf{Attachment.extension}, obrigatório e não nulo.
    \end{itemize}
    \item Category: Mapeamento direto para {\textbf{Category}}
    \begin{itemize}
        \item name: Mapeamento direto para \textbf{Category.name}, obrigatório e não nulo, é unico juntamente com \textbf{Category.item\_type}, assim como chave primária juntamente com ele. 
        \item item\_type:Mapeamento direto para \textbf{Category.item\_type}, obrigatório e não nulo, é unico juntamente com \textbf{Category.name}, assim como chave primária, juntamente com ele.
    \end{itemize}
\end{itemize}
\break
Relacionamentos:
\begin{itemize}
    \item PaymentMethod: Mapeamento de relacionamento por colunas adicionais, pois é um relacionamento $1$:$N$. Com a chave estrangeira em CreditCard..
    \item Review: Mapeamento de relacionamento por tabela própria, pois é um relacionamento $N$:$M$.
    \begin{itemize}
        \item {\textit{user\_email}}: Chave primária do usuário, por ser uma tabela de relação que liga usuário a itens. Mapeado para {\textbf{Review.user\_email}}, obrigatório e chave estrangeira.
        \item {\textit{item\_id}}: Chave primária do item, pelo mesmo motivo do atributo user\_email. Mapeado para {\textbf{Review.item\_id}}, obrigatório e chave estrangeira.
        \item comment: Mapeamento direto para {\textbf{Review.comment}}, opcional.
        \item date: Mapeamento direto para {\textbf{Review.date}}, obrigatório.
        \item rating: Mapeamento direto para {\textbf{Review.rating}}, obrigatório.
    \end{itemize}
    \item WishList: Mapeamento de relacionamento por tabela própria, pois é um relacionamento $N$:$M$. 
    \begin{itemize}
        \item {\textit{user\_email}}: Chave primária do usuário, por ser uma tabela de relação que liga usuário a itens. Mapeado para {\textbf{WhishList.user\_email}}, obrigatório e chave estrangeira.
        \item {\textit{item\_id}}: Chave primária do item, pelo mesmo motivo do atributo user\_email. Mapeado para {\textbf{WishList.item\_id}}, obrigatório e chave estrangeira.
        \item date: Mapeamento direto para  {\textbf{WhishList.date}}, obrigatório.
    \end{itemize}
    \item Download: Mapeamento de relacionamento por tabela própria, pois é um relacionamento ternário. Foi implementada uma tabela com referência (chave estrangeira) às chaves primárias de User, CreditCart e Item.
    \begin{itemize}
        \item {\textit{item\_id}}: Chave primária do item, que está fazendo parte da chave composta que identifica o prórpio Download. Mapeado para {\textbf{Download.item\_id}}, obrigatório e chave estrangeira.
        \item {\textit{user\_email}}: Chave primária do usuário, mapeado para {\textbf{WhishList.user\_email}}, obrigatório e chave estrangeira.
        \item first\_downloaded\_date: Mapeamento direto para  {\textbf{Download. \\ first\_downloaded\_date}}, obrigatório.
        \item last\_downloaded\_date: Mapeamento direto para  {\textbf{Download. \\ last\_downloaded\_date}}, obrigatório.
        \item downloaded: Mapeamento direto para  {\textbf{Download.downloaded}}, obrigatório.
    \end{itemize}
    \item Development: Mapeamento de relacionamento por colunas adicionais, pois é um relacionamento $1$:$N$. Com a chave estrangeira em App.
    \item Categorization: Mapeamento de relacionamento por tabela própria, pois é um relacionamento $N$:$M$.
    \begin{itemize}
        \item {\textit{item\_id}}: Chave estrangeira para Item e que está fazendo parte da chave primária composta. Mapeada para {\textit{Categorization.item\_id}}, obrigatória.
        \item {\textit{category\_name}} e {\textit{category\_item\_type}}: Chave estrangeira composta de Category, onde category\_name faz parte da chave composta junto com item\_id em Categorization. Mapeados respectivamente para {\textbf{Categorization.category\_name}} e {\textbf{Category.category\_item\_type}}
    \end{itemize}
    \item Attaching: Mapeamento de relacionamento por colunas adicionais, pois é um relacionamento $1$:$N$. Com a chave estrangeira em Attachment.
    \item Book\_Authorship: Mapeamento de relacionamento por colunas adicionais, pois é um relacionamento $1$:$N$. Com a chave estrangeira em Book.
    \item Book\_Language: Mapeamento de relacionamento por colunas adicionais, pois é um relacionamento $1$:$N$. Com a chave estrangeira em Book.
    \item Movie\_Language: Mapeamento de relacionamento por tabela própria, pois é um relacionamento $N$:$M$.
    \begin{itemize}
        \item type: Mapeamento direto para {\textbf{Movie\_Language.type}}, obrigatório e parte da chave primária composta.
        \item {\textit{language\_name}}: Chave estrangeira para Language e que está fazendo parte da chave primária composta. Mapeada para {\textbf{Movie\_Language.language\_name}}, obrigatória.
        \item {\textit{movie\_id}}: Chave estrangeira para Movie e que está fazendo parte da chave primária composta. Mapeada para {\textbf{Movie\_Language.movie\_id}}, obrigatória.
    \end{itemize}
    \item Movie\_Cast: Mapeamento de relacionamento por tabela própria, pois é um relacionamento $N$:$M$.
    \begin{itemize}
        \item {\textit{cast\_name}}: Chave estrangeira para Language e que está fazendo parte da chave primária composta. Mapeada para {\textbf{Movie\_Cast.cast\_name}}, obrigatória.
        \item {\textit{movie\_id}}: Chave estrangeira para Movie e que está fazendo parte da chave primária composta. Mapeada para {\textbf{Movie\_Cast.movie\_id}}, obrigatória.
        \item function: Mapeamento direto para  {\textbf{Movie\_cast.function}}, obrigatório.
    \end{itemize}
    \item Album\_Track: Mapeamento de relacionamento por colunas adicionais, pois é um relacionamento $1$:$N$. Com a chave estrangeira em Track.
    \item Album\_Authorship: Mapeamento de relacionamento por colunas adicionais, pois é um relacionamento $1$:$N$. Com a chave estrangeira em Album.
    
    
%% ===================================
\clearpage
\section*{Relatório do Programa}
A linguagem de programação escolhida para desenvolver o programa foi Python\cite{Python}, juntamente com seu web framework Flask\cite{Flask}. Ele é também chamado de microframework  porque mantem um núcleo simples mas estendível. Não há uma camada de abstração do banco de dados, validação de formulários, ou qualquer outro componente onde bibliotecas de terceiros existem para prover a funcionalidade. Fazendo com que não se tornem obscuros os detalhes de conexão com a base de dados e satisfazendo um dos tópicos do enunciado.

    \item Conexão com o banco de dados: Usando a biblioteca \textit{psycopg2}, foi feita a conexão com o banco de dados, segundo o código abaixo:

    \begin{lstlisting}[language=Python]
    def get_db():
        if "db" not in g:
            print("[INFO] Creating DB connection")
            g.db = psycopg2.connect(database='GooglePlay', user="postgres", password="postgres")

        return g.db
    \end{lstlisting}
    A partir da conexão é criado um cursor:
    \begin{lstlisting}[language=Python]
    def get_cursor():
        if "cursor" not in g:
            print("[INFO] Creting cursor")
            g.cursor = get_db().cursor()

    return g.cursor
    \end{lstlisting}
    Onde ambos são únicos para cada requisição e serão reusados se forem chamados novamente.
    \item Preparação e envio das consultas ao banco de dados  e processamento de retorno das consultas: Todas as consultas no banco de dados são feitas conforme o código abaixo, onde é construída uma query SQL e feita uma chamada da função run\_query\_parameter(), que recebe como parametro uma query SQL e retorna um vetor de vetores.
    O código trata das exceções que acontecem durante a conexão com o banco de dados, retornando a exceção bem formatada, que pode ser tratada pela parte da aplicação que fez a requisição de consulta.
    \begin{lstlisting}[language=Python]
    def promotion_film_members():
        """
        Retorna o nome e a funcao dos membros de filmes que estejam na promocao. 
        """
        try:
            query = run_select_query('SELECT movie_cast.cast_name, "function" \
                                      FROM movie_with_item item \
                                      JOIN movie_cast ON (item.id = movie_cast.movie_id) \
                                      WHERE promotion_date is not null and CURRENT_TIMESTAMP(0) <= promotion_date;')
            return [{'name': obj[0], 'function': obj[1]} for obj in query]
        except (Exception, psycopg2.Error) as error:
            return {"status": "error", "error": error}
    \end{lstlisting}
    Cada vetor, dentro do vetor de vetores, representa uma linha de dados retornada da consulta feita.
    Por exemplo: O retorno recebido no código anterior é:
    [["name1", "function1"], ["name2", "function2"], ..., ["nameX", "functionX"]].
    \item Detalhes da preparação, envio e processamento de retorno dos comandos de atualização: A inserção de novos elementos é feita segundo o código abaixo, onde é recebido como parametro os dados referente a tabela que deseja inserir e é chamado o cursor junto com a função execute(). Dentro da função execute, é passada como parametro a query SQL, junto com os dados recebidos pela função. 
    \begin{lstlisting}[language=Python]
    def add_categorization(item_uuid, category_name, category_type):
        """
        Tenta adicionar uma categorization de `category_name` do tipo `category_type`
        no Item de uuid `item_uuid`
        """
        try:
            record_to_insert = (item_uuid, category_name, category_type)
            cursor = db.get_cursor()
            cursor.execute('INSERT INTO categorization VALUES (%s, %s, %s);', record_to_insert)
            db.get_db().commit()
    
            return {'row_count': cursor.rowcount, 'status': 'Record inserted successfuly into categorization table', 'error': ''}
        except (Exception, psycopg2.Error) as error:
            return {'row_count': 0, "status": "error", "error": error}
    \end{lstlisting}
    O código trata das exceções que acontecem durante a conexão com o banco de dados e a inserção da nova instância, retornando uma exceção, que pode ser tratada da mesma forma que mencionado no tópico anterior.
    
    
    

    
\end{itemize}

%% ========================
\newpage
\newpage
\begin{thebibliography}{9}
\bibitem{GooglePlay} From Google owned by Alphabet Inc., all rights reserved, 2015-2019. Available from World Wide Web: (https://play.google.com/store)
\bibitem{Google} Owned by Alphabet Inc., all rights reserved, 2015-2019. Available from World Wide Web (https://abc.xyz/)
\bibitem{RFC5322} Defined in Email address (Wikipedia). Accesed in 31/05/2019. Available from World Wide Web (https://en.wikipedia.org/wiki/Email\_address)
\bibitem{SHA512} SHA-2 criptography, as defined at Wikipedia. Acessed in 31/05/2019. Available from World Wide Web (https://en.wikipedia.org/wiki/SHA-2)
\bibitem{ISBN} International Standard Book Number. Acessed in 31/05/2019. Available from World Wide Web (http://www.isbn.bn.br/website/)
\bibitem{Python} https://www.python.org
\bibitem{Flask} http://flask.pocoo.org/
\end{thebibliography}

\end{document}
